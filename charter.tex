\documentclass[
11pt, % The default document font size, options: 10pt, 11pt, 12pt
codirector, % Uncomment to add a codirector to the title page
]{charter} 




% El títulos de la memoria, se usa en la carátula y se puede usar el cualquier lugar del documento con el comando \ttitle
\titulo{Comunicador para centrales de alarma} 

% Nombre del posgrado, se usa en la carátula y se puede usar el cualquier lugar del documento con el comando \degreename
%\posgrado{Carrera de Especialización en Sistemas Embebidos} 
\posgrado{Carrera de Especialización en Internet de las Cosas} 
%\posgrado{Carrera de Especialización en Intelegencia Artificial}
%\posgrado{Maestría en Sistemas Embebidos} 
%\posgrado{Maestría en Internet de las cosas}

% Tu nombre, se puede usar el cualquier lugar del documento con el comando \authorname
\autor{Mariano Mondani} 

% El nombre del director y co-director, se puede usar el cualquier lugar del documento con el comando \supname y \cosupname y \pertesupname y \pertecosupname
\director{Nombre del Director}
\pertenenciaDirector{pertenencia} 
% FIXME:NO IMPLEMENTADO EL CODIRECTOR ni su pertenencia
\codirector{John Doe} % para que aparezca en la portada se debe descomentar la opción codirector en el documentclass
\pertenenciaCoDirector{FIUBA}

% Nombre del cliente, quien va a aprobar los resultados del proyecto, se puede usar con el comando \clientename y \empclientename
\cliente{Claudio Bongiorno}
\empresaCliente{X-28 Alarmas}

% Nombre y pertenencia de los jurados, se pueden usar el cualquier lugar del documento con el comando \jurunoname, \jurdosname y \jurtresname y \perteunoname, \pertedosname y \pertetresname.
\juradoUno{Nombre y Apellido (1)}
\pertenenciaJurUno{pertenencia (1)} 
\juradoDos{Nombre y Apellido (2)}
\pertenenciaJurDos{pertenencia (2)}
\juradoTres{Nombre y Apellido (3)}
\pertenenciaJurTres{pertenencia (3)}
 
\fechaINICIO{30 de abril de 2021}		%Fecha de inicio de la cursada de GdP \fechaInicioName
\fechaFINALPlan{18 de junio de 2021} 	%Fecha de final de cursada de GdP
\fechaFINALTrabajo{15 de mayo de 2022}	%Fecha de defensa pública del trabajo final


\begin{document}

\maketitle
\thispagestyle{empty}
\pagebreak


\thispagestyle{empty}
{\setlength{\parskip}{0pt}
\tableofcontents{}
}
\pagebreak


\section{Registros de cambios}
\label{sec:registro}


\begin{table}[ht]
\label{tab:registro}
\centering
\begin{tabularx}{\linewidth}{@{}|c|X|c|@{}}
\hline
\rowcolor[HTML]{C0C0C0} 
Revisión & \multicolumn{1}{c|}{\cellcolor[HTML]{C0C0C0}Detalles de los cambios realizados} & Fecha      \\ \hline
0      & Creación del documento\newline Se completa hasta el punto xx inclusive &30/04/2021 \\ \hline
%1      & Se completa hasta el punto 4 inclusive                 & dd/mm/aaaa \\ \hline
%2      & Se completa hasta el punto 7 inclusive
%		  Se puede agregar algo más \newline
%		  En distintas líneas \newline
%		  Así                                                    & dd/mm/aaaa \\ \hline
%3      & Se completa hasta el punto 11 inclusive                & dd/mm/aaaa \\ \hline
%4      & Se completa el plan	                                 & dd/mm/aaaa \\ \hline
\end{tabularx}
\end{table}

\pagebreak



\section{Acta de constitución del proyecto}
\label{sec:acta}

\begin{flushright}
Buenos Aires, \fechaInicioName
\end{flushright}

\vspace{2cm}

Por medio de la presente se acuerda con el Ing. \authorname\hspace{1px} que su Trabajo Final de la \degreename\hspace{1px} se titulará ``\ttitle'', consistirá esencialmente en la implementación de un sistema que permitirá comunicar una alarma domiciliaria con una aplicación móvil, y tendrá un presupuesto preliminar estimado de 600 hs de trabajo, con fecha de inicio \fechaInicioName\hspace{1px} y fecha de presentación pública \fechaFinalName.

Se adjunta a esta acta la planificación inicial.

\vfill

% Esta parte se construye sola con la información que hayan cargado en el preámbulo del documento y no debe modificarla
\begin{table}[ht]
\centering
\begin{tabular}{ccc}
\begin{tabular}[c]{@{}c@{}}Ariel Lutenberg \\ Director posgrado FIUBA\end{tabular} & \hspace{2cm} & \begin{tabular}[c]{@{}c@{}}\clientename \\ \empclientename \end{tabular} \vspace{2.5cm} \\ 
\multicolumn{3}{c}{\begin{tabular}[c]{@{}c@{}} \supname \\ Director del Trabajo Final\end{tabular}} \vspace{2.5cm} \\
%\begin{tabular}[c]{@{}c@{}}\jurunoname \\ Jurado del Trabajo Final\end{tabular}     &  & \begin{tabular}[c]{@{}c@{}}\jurdosname\\ Jurado del Trabajo Final\end{tabular}  \vspace{2.5cm}  \\
%\multicolumn{3}{c}{\begin{tabular}[c]{@{}c@{}} \jurtresname\\ Jurado del Trabajo Final\end{tabular}} \vspace{.5cm}                                                                     
\end{tabular}
\end{table}




\section{Descripción técnica-conceptual del proyecto a realizar}
\label{sec:descripcion}

%\begin{consigna}{black} 

Desde hace más de una década, es común encontrar en un sistema de alarma hogareño algún tipo de comunicador. Generalmente estos comunicadores permiten tanto recibir avisos cuando se produce un evento en la alarma y enviarle comandos para que se realice algun a acción en el sistema.

Con el paso de los años, la forma de conexión entre el usuario y su alarma ha ido cambiando. Comenzó siendo a través de la línea telefónica, para luego, pasar a llevarse a cabo mediante la red celular. 

Por muchos años los usuarios percibieron como ágil y novedosa la comunicación a través de mensajes de texto (SMS). El comunicador le enviaba un mensaje cuando se producía algún evento en la alarma y el usuario podía realizar una acción enviando comandos sencillos. Sin embargo, la comunicación mediante SMS fue quedando obsoleta en la vida cotideana y por lo tanto comenzó a ser considerado un medio poco confiable de interacción con un sistema de seguridad.

Impulsada por esta situación, la empresa comenzó a orientar sus esfuerzos a desarrollar nuevos productos que les permitan a los usuarios relacionarse con sus alarmas de una forma más sencilla, consistente y robusta.

En este contexto surge el presente proyecto: un equipo que se conecta al sistema de alarma como cualquier otro dispositivo y que es acompañado mediante una aplicación que permite conocer el estado de su alarma, recibir eventos y enviar comandos.

A pesar de que actualmente la competencia de X-28 Alarmas ofrece comunicadores compatibles con las centrales de alarma de la marca, el principal difirencial que va a ofrecer este nuevo producto es la posibilidad de acceder a una gran variedad de funcionalidades de la alarma. 

Esto se debe a que los comunicadores universales solo ofrecen un conjunto de funciones muy limitado, generalmente activar y desactivar la central, recibir notificaciones cuando el sistema está sonando y, en algunos casos, manejar una o dos cargas eléctricas. En cambio, el producto objeto del trabajo final, busca cubrir una más amplia gama de necesidades:

\begin{itemize}
	\item Asociar múltiples alarmas a un mismo usuario, con distintos niveles de acceso.
	\item Conocer el estado de cada uno de los sensores que componen la alarma.
	\item Permitir asignaler un nombre distintivo a cada: alarma, sensor, usuario y carga eléctrica.
	\item Recibir notificaciones no solo por disparos en la alarma, sino también por problemas en la red eléctrica, por activación o desactivación del sistema distinguiendo qué usuario lo hizo, por eventos personalizados, etc.
	\item Ser compatible con la línea de productos de automatización de X-28 Alarmas.
\end{itemize}

Además del equipo que va a ser conectado en la alarma y de la aplicación utilizada por el usuario, el sistema propuesto en este trabajo se completa con el desarrollo de un backend que permita la comunicación entre alarmas y usuarios. 

En la Figura \ref{fig:diagBloques} se muestra el diagrama en bloques del sistema. Como puede verse el comunicador se va a conectar a Internet mediante dos posibles vías: Wi-Fi o red celular. Al contemplar dos formas de conexión, no solo se otorga mayor robustez a la comunicación, sino también se cubre una mayor variedad de casos de uso. Este equipo podría ser usado mediante Wi-Fi en el caso en el que la cobertura celular, en el lugar de la instalación, fuera deficiente o en el caso en el que el usuario no quiera atarse a un abono mensual en relación al servicio celular. Por el contrario, podría ser utilizado a través de la red celular en aquellas situaciones en donde no hay una red Wi-Fi.

%\vspace{25px}

\begin{figure}[htpb]
\centering 
\includegraphics[width=\textwidth]{./Figuras/diagBloques.png}
\caption{Diagrama en bloques del sistema}
\label{fig:diagBloques}
\end{figure}

\vspace{25px}

Finalmente, puede verse que el backend se compone de:

\begin{itemize}
	\item En broker que se encarga de la comunicación entre las apps de los usuarios finales y los comunicadores
	\item Una API de tipo REST utilizada por las apps para: login, obtener información de la alarma, enviar comandos al comunicador, etc.
	\item Una base de datos en donde se lleva un registro de los comunicadores, los usuarios de las apps y el estado y configuración de cada uno de los sistemas de alarma.
\end{itemize}

%\end{consigna}


\section{Identificación y análisis de los interesados}
\label{sec:interesados}

%\begin{consigna}{black} 

\begin{table}[h!t]
%\caption{Identificación de los interesados}
%\label{tab:interesados}
\begin{tabularx}{\linewidth}{@{}|l|X|X|X|@{}}
\hline
\rowcolor[HTML]{C0C0C0} 
Rol           & Nombre y Apellido & Organización 	& Puesto 	\\ \hline
Auspiciante   & \clientename      &\empclientename	& Jefe de área       	\\ \hline
Cliente       & \clientename      &\empclientename	& Jefe de área       	\\ \hline
Impulsor      & \clientename      &\empclientename	& Jefe de área       	\\ \hline
Responsable   & \authorname       & FIUBA        	& Alumno 	\\ \hline
Colaboradores & Norberto Vergani \newline
			   Claudio Bongiorno \newline
			   Pablo Marchant \newline
			   Docentes CEIoT   & \empclientename \newline
			   					\empclientename \newline
			   					\empclientename \newline
			   					FIUBA             & Director \newline 
			   									  Jefe de área\newline
			   									  Jefe de área\newline
			   									  -       	\\ \hline
Orientador    & \supname	      & \pertesupname 	& Director Trabajo final \\ \hline
Usuario final & Clientes de X-28 Alarmas & -             	& -       	\\ \hline
\end{tabularx}
\end{table}


\begin{itemize}
	\item Auspiciante: buscando completar la línea de productos de seguridad comercializados por la
empresa, está interesado en que el proyecto se complete en el tiempo establecido y está dispuesto a
invertir para ayudar a lograrlo. Debido a ser una empresa que vende productos de seguridad, se pone
especial atención en mantener el prestigio y la confianza que los clientes depositan en la empresa, por
lo que es exigentes en cuanto a la calidad y confiabilidad del producto.
	\item Cliente: es exigente en cuanto a la integración de los nuevos dispositivos con la línea actual de productos. Su objetivo es que los nuevos equipos sean confiables y robustos en cuanto a su usabilidad.
	\item Colaboradores: Norberto Vergani tiene una amplia experiencia tanto en el diseño de hardware con
sistemas embebidos como con la programación del firmware. Se deben aprovechar especialmente sus
conocimientos en este último tema.\\
Claudio Bongiorno cuenta con conocimientos en el diseño de hardware y a participado en el diseño e implementación de gran parte de los productos actuales de la empresa. Sus sugerencias van a ayudar en el diseño del circuito impreso.\\
Pablo Marchant es el jefe de área de Comercio exterior, y va a ser el encargado de conseguir muchos de
los componentes necesarios para el hardware. Su colaboración va a permitir encontrar la forma de reducir los costos en los componentes electrónicos.
	\item Usuario final: los clientes de X-28 Alarmas estás acostumbrados a que los productos de la marca sean sencillos de utilizar. Además buscan que los nuevos productos que se ofrecen sean compatibles con los ya existentes, permitiéndoles agregarlos sin problemas a sus sistemas de alarma.
\end{itemize}

%\end{consigna}



\section{1. Propósito del proyecto}
\label{sec:proposito}

%\begin{consigna}{red}
El propósito de este proyecto consiste en diseñar e implementar un comunicador, una aplicación y un backend que le permita a los usuarios de las alarmas de X-28 comunicarse con sus sistemas de seguridad. Se busca lograr, por un lado, un producto que no solo sea sencillo de utilizar sino también robusto y seguro. Por otro lado, a lo largo del proyecto, se buscarán introducir prácticas de programación y periféricos de hardware que puedan ser reutilizados en otros proyectos  y por lo tanto aumenten la base de conocimiento del sector de Investigación y Desarrollo.
%\end{consigna}

\section{2. Alcance del proyecto}
\label{sec:alcance}

%\begin{consigna}{red}

El proyecto incluye:

\begin{itemize}
	\item El desarrollo de un sistema embebido capaz de conectarse a un sistema de alarma y a Internet , mediante Wi-Fi y red celular. Se incluye tanto hardware como firmware.
	\item El desarrollo del prototipo de una aplicación híbrida (utilizable en Android y en la Web) para poder controlar las alarmas
	\item El diseño del modelo de datos utilizado en la base de datos del backend
	\item El desarrollo de un broker que permita la comunicación entre las aplicaciones y los comunicadores.
	\item El desarrollo de una API REST que le permita a las aplicaciones obtener datos de las alarmas y enviarles comandos
\end{itemize}

El proyecto no incluye:

\begin{itemize}
	\item Diseño del gabinete para el comunicador.
	\item Las implementaciones particulares para que la aplicación funcione en iOS.
	\item Consideraciones de diseño para el despliegue del broker y la API en un entorno cloud, contemplando alta disponibilidad y escalamiento.
\end{itemize}

%\end{consigna}


\section{3. Supuestos del proyecto}
\label{sec:supuestos}

%\begin{consigna}{red}
Para el desarrollo del presente proyecto se supone que:

\begin{itemize}
	\item Se podrá diseñar e implementar el producto, logrando su correcto funcionamiento.
	\item No habrá dificultades para conseguir los componentes electrónicos necesarios.
	\item Se contará con los concocimientos necesarios para desarrollar la aplicación híbrida y el backend. En caso de no contar con ciertos conocimientos, se los podrá investigar y aplicar satisfactoriamente.
	\item Dentro de la lista de proyectos actualmente en curso dentro del área de Investigación y Desarrollo, se le dará la dedicación suficiente al presente proyecto y no se lo suspenderá en favor del desarrollo de otros productos.
\end{itemize}

%\end{consigna}

\section{4. Requerimientos}
\label{sec:requerimientos}

\begin{consigna}{red}
Los requerimientos deben numerarse y de ser posible estar agruparlos por afinidad, por ejemplo:

\begin{enumerate}
	\item Requerimientos funcionales
		\begin{enumerate}
			\item El sistema debe...
			\item Tal componente debe...
			\item El usuario debe poder...
		\end{enumerate}
	\item Requerimientos de documentación
		\begin{enumerate}
			\item Requerimiento 1
			\item Requerimiento 2 (prioridad menor)
		\end{enumerate}
	\item Requerimiento de testing...
	\item Requerimientos de la interfaz...
	\item Requerimientos interoperabilidad...
	\item etc...
\end{enumerate}

Leyendo los requerimientos se debe poder interpretar cómo será el proyecto y su funcionalidad.

Indicar claramente cuál es la prioridad entre los distintos requerimientos y si hay requerimientos opcionales. 

No olvidarse de que los requerimientos incluyen a las regulaciones y normas vigentes!!!

Y al escribirlos seguir las siguientes reglas:
\begin{itemize}
	\item Ser breve y conciso (nadie lee cosas largas). 
	\item Ser específico: no dejar lugar a confusiones.
	\item Expresar los requerimientos en términos que sean cuantificables y medibles.
\end{itemize}

\end{consigna}

\section{5. Historias de usuarios (\textit{Product backlog})}
\label{sec:backlog}

\begin{consigna}{red}
Descripción: En esta sección se deben incluir las historias de usuarios y su ponderación (\textit{history points}). Recordar que las historias de usuarios son descripciones cortas y simples de una característica contada desde la perspectiva de la persona que desea la nueva capacidad, generalmente un usuario o cliente del sistema. La ponderación es un número entero que representa el tamaño de la historia comparada con otras historias de similar tipo.

Se debe indicar explícitamente el criterio para calcular los \textit{story points} de cada historia
\end{consigna}

\section{6. Entregables principales del proyecto}
\label{sec:entregables}

\begin{consigna}{red}

Los entregables del proyecto son (ejemplo):

\begin{itemize}
	\item Manual de uso
	\item Diagrama de circuitos esquemáticos
	\item Código fuente del firmware
	\item Diagrama de instalación
	\item Informe final
	\item etc...
\end{itemize}

\end{consigna}

\section{7. Desglose del trabajo en tareas}
\label{sec:wbs}

\begin{consigna}{red}
El WBS debe tener relación directa o indirecta con los requerimientos.  Son todas las actividades que se harán en el proyecto para dar cumplimiento a los requerimientos. Se recomienda mostrar el WBS mediante una lista indexada:

\begin{enumerate}
\item Grupo de tareas 1
	\begin{enumerate}
	\item Tarea 1 (tantas hs)
	\item Tarea 2 (tantas hs)
	\item Tarea 3 (tantas hs)
	\end{enumerate}
\item Grupo de tareas 2
	\begin{enumerate}
	\item Tarea 1 (tantas hs)
	\item Tarea 2 (tantas hs)
	\item Tarea 3 (tantas hs)
	\end{enumerate}
\item Grupo de tareas 3
	\begin{enumerate}
	\item Tarea 1 (tantas hs)
	\item Tarea 2 (tantas hs)
	\item Tarea 3 (tantas hs)
	\item Tarea 4 (tantas hs)
	\item Tarea 5 (tantas hs)
	\end{enumerate}
\end{enumerate}

Cantidad total de horas: (tantas hs)

Se recomienda que no haya ninguna tarea que lleve más de 40 hs. 

\end{consigna}

\section{8. Diagrama de Activity On Node}
\label{sec:AoN}

\begin{consigna}{red}
Armar el AoN a partir del WBS definido en la etapa anterior. 

%La figura \ref{fig:AoN} fue elaborada con el paquete latex tikz y pueden consultar la siguiente referencia \textit{online}:

%\url{https://www.overleaf.com/learn/latex/LaTeX_Graphics_using_TikZ:_A_Tutorial_for_Beginners_(Part_3)\%E2\%80\%94Creating_Flowcharts}

\end{consigna}

\begin{figure}[htpb]
\centering 
\includegraphics[width=.8\textwidth]{./Figuras/AoN.png}
\caption{Diagrama en \textit{Activity on Node}}
\label{fig:AoN}
\end{figure}

Indicar claramente en qué unidades están expresados los tiempos.
De ser necesario indicar los caminos semicríticos y analizar sus tiempos mediante un cuadro.
Es recomendable usar colores y un cuadro indicativo describiendo qué representa cada color, como se muestra en el siguiente ejemplo:



\section{9. Diagrama de Gantt}
\label{sec:gantt}

\begin{consigna}{red}

Existen muchos programas y recursos \textit{online} para hacer diagramas de gantt, entre los cuales destacamos:

\begin{itemize}
\item Planner
\item GanttProject
\item Trello + \textit{plugins}. En el siguiente link hay un tutorial oficial: \\ \url{https://blog.trello.com/es/diagrama-de-gantt-de-un-proyecto}
\item Creately, herramienta online colaborativa. \\\url{https://creately.com/diagram/example/ieb3p3ml/LaTeX}
\item Se puede hacer en latex con el paquete \textit{pgfgantt}\\ \url{http://ctan.dcc.uchile.cl/graphics/pgf/contrib/pgfgantt/pgfgantt.pdf}
\end{itemize}

Pegar acá una captura de pantalla del diagrama de Gantt, cuidando que la letra sea suficientemente grande como para ser legible. 
Si el diagrama queda demasiado ancho, se puede pegar primero la ``tabla'' del Gantt y luego pegar la parte del diagrama de barras del diagrama de Gantt.

Configurar el software para que en la parte de la tabla muestre los códigos del EDT (WBS).\\
Configurar el software para que al lado de cada barra muestre el nombre de cada tarea.\\
Revisar que la fecha de finalización coincida con lo indicado en el Acta Constitutiva.

En la figura \ref{fig:gantt}, se muestra un ejemplo de diagrama de gantt realizado con el paquete de \textit{pgfgantt}. En la plantilla pueden ver el código que lo genera y usarlo de base para construir el propio.

\begin{figure}[htbp]
\begin{center}
\begin{ganttchart}{1}{12}
  \gantttitle{2020}{12} \\
  \gantttitlelist{1,...,12}{1} \\
  \ganttgroup{Group 1}{1}{7} \\
  \ganttbar{Task 1}{1}{2} \\
  \ganttlinkedbar{Task 2}{3}{7} \ganttnewline
  \ganttmilestone{Milestone o hito}{7} \ganttnewline
  \ganttbar{Final Task}{8}{12}
  \ganttlink{elem2}{elem3}
  \ganttlink{elem3}{elem4}
\end{ganttchart}
\end{center}
\caption{Diagrama de gantt de ejemplo}
\label{fig:gantt}
\end{figure}

\end{consigna}


\section{9. Presupuesto detallado del proyecto}
\label{sec:presupuesto}

\begin{consigna}{red}
Si el proyecto es complejo entonces separarlo en partes:
\begin{itemize}
	\item Un total global, indicando el subtotal acumulado por cada una de las áreas.
	\item El desglose detallado del subtotal de cada una de las áreas.
\end{itemize}

IMPORTANTE: No olvidarse de considerar los COSTOS INDIRECTOS.

\end{consigna}

\begin{table}[htpb]
\centering
\begin{tabularx}{\linewidth}{@{}|X|c|r|r|@{}}
\hline
\rowcolor[HTML]{C0C0C0} 
\multicolumn{4}{|c|}{\cellcolor[HTML]{C0C0C0}COSTOS DIRECTOS} \\ \hline
\rowcolor[HTML]{C0C0C0} 
Descripción &
  \multicolumn{1}{c|}{\cellcolor[HTML]{C0C0C0}Cantidad} &
  \multicolumn{1}{c|}{\cellcolor[HTML]{C0C0C0}Valor unitario} &
  \multicolumn{1}{c|}{\cellcolor[HTML]{C0C0C0}Valor total} \\ \hline
 &
  \multicolumn{1}{c|}{} &
  \multicolumn{1}{c|}{} &
  \multicolumn{1}{c|}{} \\ \hline
 &
  \multicolumn{1}{c|}{} &
  \multicolumn{1}{c|}{} &
  \multicolumn{1}{c|}{} \\ \hline
\multicolumn{1}{|l|}{} &
   &
   &
   \\ \hline
\multicolumn{1}{|l|}{} &
   &
   &
   \\ \hline
\multicolumn{3}{|c|}{SUBTOTAL} &
  \multicolumn{1}{c|}{} \\ \hline
\rowcolor[HTML]{C0C0C0} 
\multicolumn{4}{|c|}{\cellcolor[HTML]{C0C0C0}COSTOS INDIRECTOS} \\ \hline
\rowcolor[HTML]{C0C0C0} 
Descripción &
  \multicolumn{1}{c|}{\cellcolor[HTML]{C0C0C0}Cantidad} &
  \multicolumn{1}{c|}{\cellcolor[HTML]{C0C0C0}Valor unitario} &
  \multicolumn{1}{c|}{\cellcolor[HTML]{C0C0C0}Valor total} \\ \hline
\multicolumn{1}{|l|}{} &
   &
   &
   \\ \hline
\multicolumn{1}{|l|}{} &
   &
   &
   \\ \hline
\multicolumn{1}{|l|}{} &
   &
   &
   \\ \hline
\multicolumn{3}{|c|}{SUBTOTAL} &
  \multicolumn{1}{c|}{} \\ \hline
\rowcolor[HTML]{C0C0C0}
\multicolumn{3}{|c|}{TOTAL} &
   \\ \hline
\end{tabularx}%
\end{table}


\section{10. Matriz de asignación de responsabilidades}
\label{sec:responsabilidades}
\begin{consigna}{red}
Establecer la matriz de asignación de responsabilidades y el manejo de la autoridad completando la siguiente tabla:

\begin{table}[htpb]
\centering
\resizebox{\textwidth}{!}{%
\begin{tabular}{|c|c|c|c|c|c|}
\hline
\rowcolor[HTML]{C0C0C0} 
\cellcolor[HTML]{C0C0C0} &
  \cellcolor[HTML]{C0C0C0} &
  \multicolumn{4}{c|}{\cellcolor[HTML]{C0C0C0}Listar todos los nombres y roles del proyecto} \\ \cline{3-6} 
\rowcolor[HTML]{C0C0C0} 
\cellcolor[HTML]{C0C0C0} &
  \cellcolor[HTML]{C0C0C0} &
  Responsable &
  Orientador &
  Equipo &
  Cliente \\ \cline{3-6} 
\rowcolor[HTML]{C0C0C0} 
\multirow{-3}{*}{\cellcolor[HTML]{C0C0C0}\begin{tabular}[c]{@{}c@{}}Código\\ WBS\end{tabular}} &
  \multirow{-3}{*}{\cellcolor[HTML]{C0C0C0}Nombre de la tarea} &
  \authorname &
  \supname &
  Nombre de alguien &
  \clientename \\ \hline
 &  &  &  &  &  \\ \hline
 &  &  &  &  &  \\ \hline
 &  &  &  &  &  \\ \hline
\end{tabular}%
}
\end{table}

{\footnotesize
Referencias:
\begin{itemize}
	\item P = Responsabilidad Primaria
	\item S = Responsabilidad Secundaria
	\item A = Aprobación
	\item I = Informado
	\item C = Consultado
\end{itemize}
} %footnotesize

Una de las columnas debe ser para el Director, ya que se supone que participará en el proyecto.
A su vez se debe cuidar que no queden muchas tareas seguidas sin ``A'' o ``I''.

Importante: es redundante poner ``I/A'' o ``I/C'', porque para aprobarlo o responder consultas primero la persona debe ser informada.

\end{consigna}

\section{11. Gestión de riesgos}
\label{sec:riesgos}

\begin{consigna}{red}
a) Identificación de los riesgos (al menos cinco) y estimación de sus consecuencias:
 
Riesgo 1: detallar el riesgo (riesgo es algo que si ocurre altera los planes previstos de forma negativa)
\begin{itemize}
	\item Severidad (S): mientras más severo, más alto es el número (usar números del 1 al 10).\\
	Justificar el motivo por el cual se asigna determinado número de severidad (S).
	\item Probabilidad de ocurrencia (O): mientras más probable, más alto es el número (usar del 1 al 10).\\
	Justificar el motivo por el cual se asigna determinado número de (O). 
\end{itemize}   

Riesgo 2:
\begin{itemize}
	\item Severidad (S): 
	\item Ocurrencia (O):
\end{itemize}

Riesgo 3:
\begin{itemize}
	\item Severidad (S): 
	\item Ocurrencia (O):
\end{itemize}


b) Tabla de gestión de riesgos:      (El RPN se calcula como RPN=SxO)

\begin{table}[htpb]
\centering
\begin{tabularx}{\linewidth}{@{}|X|c|c|c|c|c|c|@{}}
\hline
\rowcolor[HTML]{C0C0C0} 
Riesgo & S & O & RPN & S* & O* & RPN* \\ \hline
       &   &   &     &    &    &      \\ \hline
       &   &   &     &    &    &      \\ \hline
       &   &   &     &    &    &      \\ \hline
       &   &   &     &    &    &      \\ \hline
       &   &   &     &    &    &      \\ \hline
\end{tabularx}%
\end{table}

Criterio adoptado: 
Se tomarán medidas de mitigación en los riesgos cuyos números de RPN sean mayores a...

Nota: los valores marcados con (*) en la tabla corresponden luego de haber aplicado la mitigación.

c) Plan de mitigación de los riesgos que originalmente excedían el RPN máximo establecido:
 
Riesgo 1: plan de mitigación (si por el RPN fuera necesario elaborar un plan de mitigación).
  Nueva asignación de S y O, con su respectiva justificación:
  - Severidad (S): mientras más severo, más alto es el número (usar números del 1 al 10).
          Justificar el motivo por el cual se asigna determinado número de severidad (S).
  - Probabilidad de ocurrencia (O): mientras más probable, más alto es el número (usar del 1 al 10).
          Justificar el motivo por el cual se asigna determinado número de (O).

Riesgo 2: plan de mitigación (si por el RPN fuera necesario elaborar un plan de mitigación).
 
Riesgo 3: plan de mitigación (si por el RPN fuera necesario elaborar un plan de mitigación).

\end{consigna}


\section{12. Gestión de la calidad}
\label{sec:calidad}

\begin{consigna}{red}
Para cada uno de los requerimientos del proyecto indique:
\begin{itemize} 
\item Req \#1: copiar acá el requerimiento.

\begin{itemize}
	\item Verificación para confirmar si se cumplió con lo requerido antes de mostrar el sistema al cliente. Detallar 
	\item Validación con el cliente para confirmar que está de acuerdo en que se cumplió con lo requerido. Detallar  
\end{itemize}

\end{itemize}

Tener en cuenta que en este contexto se pueden mencionar simulaciones, cálculos, revisión de hojas de datos, consulta con expertos, mediciones, etc.  Las acciones de verificación suelen considerar al entregable como ``caja blanca'', es decir se conoce en profundidad su funcionamiento interno.  En cambio, las acciones de validación suelen considerar al entregable como ``caja negra'', es decir, que no se conocen los detalles de su funcionamiento interno.

\end{consigna}

\section{13. Comunicación del proyecto}
\label{sec:comunicaciones}

El plan de comunicación del proyecto es el siguiente:

\begin{table}[htpb]
\centering
\begin{tabularx}{\linewidth}{@{}|X|C{2.4cm}|C{3cm}|C{1.8cm}|C{2cm}|C{2.1cm}|@{}}
\hline
\rowcolor[HTML]{C0C0C0} 
\multicolumn{6}{|c|}{\cellcolor[HTML]{C0C0C0}PLAN DE COMUNICACIÓN DEL PROYECTO}           \\ \hline
\rowcolor[HTML]{C0C0C0} 
¿Qué comunicar? & Audiencia & Propósito & Frecuencia & Método de comunicac. & Responsable \\ \hline
                &           &           &            &                      &             \\ \hline
                &           &           &            &                      &             \\ \hline
                &           &           &            &                      &             \\ \hline
                &           &           &            &                      &             \\ \hline
                &           &           &            &                      &             \\ \hline
\end{tabularx}
\end{table}



\section{14. Procesos de cierre}    
\label{sec:cierre}

\begin{consigna}{red}
Establecer las pautas de trabajo para realizar una reunión final de evaluación del proyecto, tal que contemple las siguientes actividades:

\begin{itemize}
	\item Pautas de trabajo que se seguirán para analizar si se respetó el Plan de Proyecto original:
	 - Indicar quién se ocupará de hacer esto y cuál será el procedimiento a aplicar. 
	\item Identificación de las técnicas y procedimientos útiles e inútiles que se emplearon, y los problemas que surgieron y cómo se solucionaron:
	 - Indicar quién se ocupará de hacer esto y cuál será el procedimiento para dejar registro.
	\item Indicar quién organizará el acto de agradecimiento a todos los interesados, y en especial al equipo de trabajo y colaboradores:
	  - Indicar esto y quién financiará los gastos correspondientes.
\end{itemize}

\end{consigna}


\end{document}
